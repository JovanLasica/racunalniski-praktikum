\documentclass{article}

\usepackage[T1]{fontenc}
\usepackage[utf8]{inputenc}
\usepackage[slovene]{babel}

% Ker imajo nekateri citati povezave
\usepackage{hyperref}

% Za izdelavo stvarnega kazala
\usepackage{makeidx}

% Ukaz, ki naredi stvarno kazalo, a ga ne prikaže (mora biti v preambuli)
\makeindex

\begin{document}

\title{Kako citiramo literaturo\\ in naredimo stvarno kazalo}
\author{Andrej Bauer}
\maketitle

\begin{abstract}
  V tem članku bomo citirali vire iz literature in medmrežja.
  Prvi nauk je, da se v povzetku nikoli ne citira literature,
  ker je povzetek pogosto objavljen ločeno od članka.

  Potem bomo pogledali še, kako se naredi stvarno kazalo.
\end{abstract}


\section{Kaj in kako citiramo?}

Ko pišemo znanstveno ali strokovno besedilo, uporabljamo razne vire. Le-te je treba vedno
navesti. Pravimo, da jih \emph{citiramo}. V {\LaTeX}u to naredimo z Bibtexom, ki je
samostojen program za upravljanje z bibliografskimi podatki.

V znanstveni literaturi velja, da lahko kot zanesljiv vir uporabimo \emph{recenzirana
  dela}. Mednje štejemo predvsek članeke iz znanstvenih revij in nekaterih znanstvenih
konferenc ter monografij z recenzijo. Seveda lahko navajamo tudi druge vire (doktorske
disertacije, ``preprint'', članke s konferenc brez recenzije ipd.), a ti so pogosto manj
zanesljivi kot recenzirani članki.
\index{prostor!evklidski}%
Na sploh moramo biti kritični in pozorni, ko se
zanašamo na delo drugih. Mimogrede, Wikipedia je polna napak.

\section{Program Bibtex}

Bibliografijo damo v datoteko s končnico~\verb|.bib|. Uporabimo jo z ukazom
\verb|\bibliography|, glej konec te datoteke. Z ukazom \verb|cite| citiramo posamezne
enote~\cite{bauer06}, lahko več naenkrat~\cite{ersov80,lesnik10}. Če želimo citirati točno
določen izrek v članku~\cite[Izrek~2.1]{lawvere69}, lahko naredimo tudi to.
\index{prostor!topološki}%

\section{Stvarno kazalo}

Stvarno kazalo%
\index{stvarno kazalo}%
je malo bolj komplicirana reč. V člankih ga običajno ne uporabljamo, za daljša besedila pa
pride prav. V tej datoteki vidite osnovno uporabo paketa \texttt{makeidx} za izdelavo
stvarnega kazala.
\index{prostor}%

Če želite videti resno stvarno kazalo, si na primer oglejte knjigo ``Homotopy type theory:
Univalent foundations of mathematics''~\cite{hottbook}.



%% BIBLIOGRAFIJA
%% Najprej nastavimo stil bibliografije (plain, alpha, abbrv, ...)
\bibliographystyle{plain}

%% Datoteka za Bibtex
\bibliography{literatura}

%% STVARNO KAZALO
\printindex

\end{document}
