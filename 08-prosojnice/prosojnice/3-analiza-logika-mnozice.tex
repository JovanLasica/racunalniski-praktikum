\begin{frame}{Logika in množice}
	\begin{enumerate}
		\item
		Poišči preneksno obliko formule
		\( \exists x: P(x) \land \forall x: Q(x) \Rightarrow \forall x: R(X) \).
		\item 
		Definiramo množici A=[2,~5] in B=\{0,~1,~2,~3,~4\dots\}.
		V ravnino nariši:
		\begin{enumerate}
		   \item \( A \cap B \times \emptyset \)
		   \item \( (A \cup B)^c \times \mathbb{R} \)
		\end{enumerate}
		\item
		Dokaži:
		\begin{itemize}
			\item \( (A \Rightarrow B) \sim (\neg B \Rightarrow \neg A) \)
			\item \( \neg (A \lor B) \sim \neg A \land \neg B \)
		\end{itemize}
	\end{enumerate}
\end{frame}

\begin{frame}{Analiza}
	\begin{enumerate}
		\item Pokaži, da je funkcija \( x \mapsto \sqrt{x} \) enakomerno zvezna na $[0, \infty)$.
		\item Katero krivuljo določa sledeč parametričen zapis?
		% Spodaj si pomagajte z dokumentacijo o razmikih v matematičnem načinu.
		% https://www.overleaf.com/learn/latex/Spacing_in_math_mode
		$$
		   x(t) = a \cos t, \qquad % tu manjka ukaz za presledek
		   y(t) = b \sin t, \qquad % tu manjka ukaz za presledek
		   t \in [0, 2 \pi]
		$$ 
		\item
		Pokaži, da ima \( f(x)=3x+sin(2x) \) inverzno funkcijo in izračunaj \( (f^{-1})'(3\pi) \).
		
		\item
		Izračunaj integral 
		% V rešitvah smo spodnji integral zapisali v vrstičnem načinu,
		% ampak v prikaznem slogu. To naredite tako, da v matematičnem načinu najprej
		% uporabite ukaz displaystyle.
		% Pred dx je presledek: pravi ukaz je \,
		\( \displaystyle \int\frac{2+\sqrt{x+1}}{(x+1)^2-\sqrt{x+1}}\, dx \)
		\item 
		Naj bo $g$ zvezna funkcija. Ali posplošeni integral 
		\( \int_{0}^{1}\frac{g(x)}{x^2}\, dx \)
		konvergira ali divergira? Utemelji.
	\end{enumerate}
\end{frame}

\begin{frame}{Kompleksna števila}
	\begin{enumerate}
		\item
		Naj bo $z$ kompleksno število, $z \ne 1$ in $|z|=1$  .
		Dokaži, da je število \( i \, \frac{z+1}{z-1} \) realno.
		\item
		Poenostavi izraz:
		\large
		\begin{align*}
		\frac{\dfrac{3+i}{2-2i}+\dfrac{7i}{1-i}}{1+\dfrac{i-1}{4}-\dfrac{5}{2-3i}}
		\end{align*}
	\end{enumerate}
\end{frame}